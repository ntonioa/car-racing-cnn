\subsection{Hyperparameter fine-tuning}
After the the training and evaluation phases are over, we may consider how satisfactory the results are and whether we can improve them by tuning the hyperparameters. There are lots of hyperparameters that can be fine-tuned, and as a matter of fact the ones in the models described in Sec. \ref{sec:cnn_tuning_fitting} and shown in Fig. \ref{fig:model_structures} are not chosen randomly, but they are the result of a preliminary tuning phase based on trial and error that is not reported here for the sake of brevity.
\\Instead, we are only going to focus on the analysis of the hyperparameters whose effect is more evident (see Sec. \ref{sec:effect_of_class_weights}), namely the weights of the classes used during training, by relating them to the aforementioned performance metrics.